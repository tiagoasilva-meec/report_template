%MPEG1/2 layers I/II encoder using a RISC-V processor and hardware accelerators

%This section, or section group, should explain in detail the problem this work is trying to solve and why it is essential. More sections containing background or state-of-the-art descriptions may be added if that improves the explanation.
%Describe the existing attempts at solving the problem and their limitations. One must perform a bibliography search, identify the principal works on this topic and reference them in this document. An example citation is given in this sentence [1].

\subsection{MPEG-1/2 layers I/II}

The Moving Picture Experts Group (MPEG) is a working group that sets standards for media coding, established by International Organization for Standardization (ISO)~\cite{iso} and International Electrotechnical Commission (IEC)~\cite{iec}. Thus, the MPEG standard is a set of specifications for audio and video compression, containing two variations (among others), MPEG-1 and MPEG-2. Both variations cover audio and video, but since this work is focused on audio, the relevant standards are MPEG-1 Audio and MPEG-2 Audio.
%is a working group of ISO/IEC with the mission to develop standards for coded representation of digital audio and video and related data.

The MPEG-1 Audio, defined by ISO/IEC 11172-3 (explained in the next section), standardizes the information that an audio encoder must produce to write a bitstream conformant with the standard requirements. Moreover, it standardizes how an audio decoder has to parse, decompress, and resynthesize the information to reconstruct the original audio stream.

This standard performs perceptual audio coding, which does not attempt to retain the input signal exactly after encoding and decoding. Instead, it ensures that the output signal sounds the same to a human listener. %, by eliminating the parts of the input signal that are irrelevant to the human ear 
More precisely, an MPEG-1 audio encoder transforms the sound signal into the frequency domain, eliminates the frequency components that are masked by stronger frequency components, and packages the analyzed signal into a compressed audio bitstream.

%working principle
Focusing on the encoding process, the primary psychoacoustic effect is called "auditory masking", where parts of a signal are not audible due to the function of the human auditory system. For example, if there is a sound that consists mainly of one frequency, all other sounds that consist of a close frequency but are much quieter will not be heard. \\
%where exploiting the limitations of perception and removal of irrelevance is key to achieving a significant reduction in bitrate while preserving subjective audio quality.
Considering this, the parts of the signal that are masked are commonly called irrelevant, as opposed to the redundant parts which are removed. % by a lossless coding operation 
To eliminate this irrelevancy, the encoder contains a \textbf{psychoacoustic model} which analyzes the input signals within consecutive time blocks and determines, for each block, the spectral components of the input audio signal (by applying a \textbf{frequency transform}). Then, it models the masking properties of the human auditory system, and estimates the noise level for each frequency band, usually called "threshold of masking". 
In parallel, the input signal is fed through a time-to-frequency mapping, resulting in spectrum components for subsequent coding. \\
Finally, in the \textbf{quantization and coding} stage, the encoder tries to allocate the available number of data bits, meeting both the bitrate and masking requirements ("threshold of masking"). The information on how the bits are distributed over the spectrum is contained in the bitstream as side information.

%layers
The MPEG-1 standardizes three different coding schemes, namely Layer I, Layer II, and Layer III, with the first two layers being the relevant ones in this work.
Layer I has the lowest complexity and is specifically suitable for applications where the encoder complexity plays an important role.
Layer II requires a more complex encoder and decoder, being directed towards one-to-many applications, i.e. one encoder serves many decoders. 

Compared to Layer I, Layer II can remove more of the signal redundancy and apply the psychoacoustic threshold more efficiently.
%bitstream
In Layer II, the digitized audio signal is divided up into blocks of 1152 samples, with each block being encoded within one MPEG-1 audio frame.
Therefore, an MPEG-1 audio stream consists of consecutive audio frames, each one containing a header and the encoded data. The header contains general information, such as MPEG Layer, sampling frequency, number of channels, etc. Although most of this information may be the same for all frames, MPEG decided to give each audio frame a header to simplify synchronization and bitstream editing.

%MPEG-2
All the previous information describes one variation of MPEG, MPEG-1 Audio. This variation represents the first phase of dealing with mono and two-channel stereo sound coding, at sampling frequencies commonly used for high-quality audio (48, 44.1, and 32 kHz).\\
In addition, there is a second variation, MPEG-2 Audio, which includes three main points.
The first point is the extension of MPEG-1 to lower sampling frequencies (16 kHz, 22.05 kHz, and 24 kHz), providing better sound quality at very low bit rates.
The second point is the backward-compatible (BC) extension of MPEG-1 to multichannel sound, supporting up to 5 full bandwidth channels plus one low-frequency enhancement channel. The MPEG-2 BC stream adheres to the structure of a MPEG-1 bitstream, meaning that a MPEG-2 BC stream can be read and interpreted by a MPEG-1 audio decoder.
The third and last point is a new coding scheme called Advanced Audio Coding (AAC), which is more efficient and presents higher quality.

%applications
Today, the MPEG-1 Audio standard is the most widely compatible lossy audio format in the world, thanks to technical merits and excellent audio quality performance.
Within the professional and consumer market, four fields of applications can be identified, namely broadcasting, storage, multimedia, and telecommunication. %This variety of applications is possible because of the big range of bitrates and the numerous configurations allowed in MPEG-1. \\
%what is the work? motivation?

\subsection{IP core}

Knowing the wide range of customers that need digital audio, belonging to all industries, the goal of this work is to develop an IP core to encode MPEG-1/2 Layer I/II, using a RISC-V processor and hardware accelerators.

An intellectual property core (IP core) consists in a block of logic or data that is used in a semiconductor chip when making a field-programmable gate array (FPGA)~\cite{fpga} or application-specific integrated circuit (ASIC)~\cite{asic}.
Therefore, IP cores are usually the property of a particular person or company, being created throughout the design process and eventually turned into components for reuse. Third-party IPs can also be purchased and implemented into designs. 

Ideally, an IP core should be entirely portable, meaning it should be possible to insert it into any vendor technology or design methodology. However, this is not always the case, existing two main categories of IP cores, soft IP core, and hard IP core.\\
A soft IP core is generally offered as a synthesizable RTL model. It is developed in a hardware description language like SystemVerilog~\cite{ieee:systemVerilog} or VHSIC Hardware Description Language (VHDL)~\cite{ieee:vhdl}, or can occasionally be provided synthesized with a gate-level netlist. One advantage of this IP is the possibility to customize during the physical design phase and map to any process technology.\\
A hard IP core has logic and physical implementation, meaning that its physical layout is finished and fixed in a particular process technology.
One advantage of this core is the better predictability of chip timing performance and area for its technology. 

A company that purchases an IP core license usually receives everything that's required to design, test, and implement the core in its product. It may also receive logic and test patterns, signal specifications, design notes, and a list of known bugs or limitations.

%introduction for the examples
Two IP Cores, two Chips, and one Software that perform MPEG-1/2 Layer I/II audio encoding are presented in the following subsections.

\subsection{IP CORE: CWda74 MPEG-1/2 – Layer I/II Audio Encoder}

The \textit{CWda74}~\cite{CWda74} is an audio IP core capable of encoding one audio stream in real-time, provided by \textit{Coreworks, S.A.}~\cite{coreworks}.

This IP core contains the MPEG-1/2 Layer I/II encoder software and the \textit{Coreworks} processor-based hardware audio engine platform (\textit{CWda1011}).\\
Initially, the software is compiled into a binary file, which can be automatically boot-loaded through one of the control interfaces, parallel Advanced Microcontroller Bus Architecture (AMBA) Advanced Peripheral Bus (APB) or serial Serial Peripheral Interface (SPI). 
Once the software is loaded, the program runs on the audio engine platform. The system can be configured, controlled, and monitored through a configuration, control, and status register file, accessed by the control interfaces. \\
The Audio Input and Output Interfaces use a native parallel interface. Other standard audio interfaces, such as Inter-IC Sound (I2S)/Time Division Multiplexed (TDM) and Sony/Philips Digital Interface (SPDIF), are also available.
The Memory Interface can be AMBA Advanced eXtensible Interface (AXI) (for ASIC or Xilinx~\cite{xilinx} FPGA), Avalon (for Altera~\cite{intel} FPGA), or Memory Interface Generator (MIG) (for Xilinx FPGA).

The \textit{CWda74} IP core delivers Program binary, Software manual, Netlist or RTL, Implementation constraints, and Hardware datasheet.\\
As attributes, it presents low operation frequency and low power consumption, with the possibility of being optimized to fulfill different design specifications.
Table~\ref{tab:coreworks} describes the main features.

\vspace{0.5cm}

\begin{table}[h]
    \centering
    \begin{tabular}{|c|}
        \hline
        \textbf{Features} \\
        \hline
         ISO/IEC 11172-3 and 13818-3 standards \\
         \hline
         Fraunhofer IIS high-quality software\\
         \hline
         Mono, dual mono, stereo, and joint stereo channel modes \\
         \hline
         16, 22.05, 24, 32, 44.1, and 48 kHz sampling rates \\
         \hline
         16 to 24-bit input audio resolution \\
         \hline
         300 kB external memory requirement\\
         \hline
         Configurable output latency \\
         \hline
         1 frame minimum latency\\
         \hline
         Control, configuration, and monitoring protocol \\
         \hline
         Real-time operation @75 MHz \\
         \hline
    \end{tabular}
    \caption{Table of \textit{CWda74} features.}
    \label{tab:coreworks}
\end{table}

\vspace{0.5cm}

\subsection{IP CORE: IPB-MPEG-SE MPEG-1/2 – LAYER I/II AUDIO ENCODER}

The \textit{IPB-MPEG-SE}~\cite{ipb-mpeg-se} is an audio IP core capable of encoding up to two stereo audio streams in real-time, provided by \textit{IPbloq}~\cite{ipbloq}.

This IP core is designed to run on the \textit{IPbloq} audio engine platform \textit{IPB-PLAT}, which supports the encoding and decoding of multiple streams in multiple formats, on a single device. More precisely, the \textit{IPB-MPEG-SE} software requires an instance of the \textit{IPB-PLAT} audio engine platform with only one processor. 

Initially, the program is uploaded using a hardware interface. Then, the system is configured, run, and monitored through a configuration, control, and status register file, accessed by the same Control Interface.
The Audio Input and Output Interfaces include a native parallel interface. Other interfaces, such as I2S/TDM and SPDIF/Audio Engineering Society 3 (AES3), are also available.

The \textit{IPB-MPEG-SE} IP core delivers Program binary, Software manual, RTL of FPGA netlist, Implementation constraints, and Hardware datasheet.\\
As attributes, it presents low operation frequency, low power consumption, and compact hardware implementation, fitting economically in FPGAs and ASICs.
Table~\ref{tab:ipbloq} describes the main features.

\vspace{0.5cm}

\begin{table}[h]
    \centering
    \begin{tabular}{|c|}
        \hline
        \textbf{Features} \\
        \hline
         ISO/IEC 11172-3 and 13818-3 standards\\
         \hline
         Fraunhofer IIS high-quality software \\
         \hline
         Mono, dual mono, stereo, and joint stereo channel modes\\
         \hline
         16, 22.05, 24, 32, 44.1, and 48 kHz sampling rates \\
         \hline
         16 to 24-bit input audio resolution \\
         \hline
         300 kB external memory requirement\\
         \hline
         Configurable output latency \\
         \hline
         1 frame minimum latency\\
         \hline
         Control, configuration, and monitoring protocol\\
         \hline
         Real-time operation @150 MHz for two audio streams\\
         \hline
    \end{tabular}
    \caption{Table of \textit{IPB-MPEG-SE} features.}
    \label{tab:ipbloq}
\end{table}

%\mbox{}
\vspace{5mm}
%\mbox{}

\subsection{CHIP: CX23415 MPEG-2 Codec}

The \textit{CX23415}~\cite{cx23415} is a low-cost, full-duplex MPEG-2 codec that integrates the functionality of several Integrated Circuits (ICs) in a single device, provided by \textit{Conexant Systems, Inc}~\cite{conexant}.

This chip was the first device to deliver MPEG-2 audio/video encoding and decoding, transport stream (TS) generation, and on-screen display control in a single chip. The ability to incorporate up to five different chip functionalities allowed for reducing the cost of designing and manufacturing digital audio and video products.

%As attributes, the CX23415 presents prefiltering improvements, including built-in linear filters that dynamically change or soften images in the pre-processing stage. As a result, users obtain the best possible picture, even as the data rate is reduced and the recording time is lengthened. Other video quality improvements included an increased motion search range, the decoupling of motion estimation from encoders, and an adaptive quantization scheme. 

%The CX23415 incorporates an OSD for graphics control and advanced GUI acceleration to support user interfaces and the display containing broadcast and service information from electronic program guides. The controller includes a BITBLT acceleration engine and Deflicker filter, supporting a variety of pixel formats, including an 8-bit color index and 32-bit RGBA 8:8:8:8.

For audio encoding and decoding, the \textit{CX23415} integrates MPEG-1 Layer II, with sampling rates of 32 kHz, 44.1 kHz, and 48 kHz, and compressed bit rates up to 448 kbit/sec. The encoder supports 16-bit samples, while the decoder supports 16-, 18-, or 20-bit outputs.

As audio input and output, this chip supports Stereo Sony I2S. As MPEG input and output, it supports Peripheral Component Interconnect Direct memory access (PCI DMA) master or PCI slave, 8-bit parallel program data, 8-bit parallel SPI transport data, and 1-bit serial transport data.

The \textit{CX23415} most relevant features are high-quality real-time encoding and MPEG-1 and MPEG-2 support.

\subsection{CHIP: MPEG-2 Encoder CW-4888}

The \textit{CW-4888}~\cite{cw4888} is a MPEG-2 encoder that feeds video signals of analog program sources, like cameras and broadcasters, to digital broadcast networks.

Initially, this device receives the standard composite video and the associated sound signals. Then, it digitizes and compresses the input according to the MPEG-2 standard, outputting the result as Asynchronous Serial Interface (ASI)~\cite{asi} or Internet Protocol (IP)~\cite{ip} streams.

For audio, this device supports mono, dual, stereo, and joint stereo sound modes. The audio input signal is converted by a dual-channel encoder, which performs MPEG-1 layer I/II compression based on ISO/IEC 11172-3. 
The bit rate can be set between 32 and 448 kbit/s, with sampling frequencies of 33 kHz, 44.1 kHz, and 48 kHz.

The \textit{CW-4888}'s most relevant features are FPGA circuitry and the option for 2 or 4 independent encoder units in one frame.
As attributes, it presents extremely low power consumption, high reliability, and a long lifespan.

\subsection{CHIP: Futura II ASI+IP™}

The \textit{Futura II}~\cite{futura} is a Broadcast oriented MPEG-2/H.264 encoder that supports all standard broadcast formats, including North American standards. 

%delivers MPEG-2 video at bit rates of 3.4 to 19.39 Mbps and H.264 video at 2.5 to 19.39 Mbps.
This device, developed by \textit{Magnum Semiconductor Inc.}, is capable of encoding MPEG-1 Layer II at 192, 224, 256, 320, and 384 Kbps, with sampling rates of 32, 44.1, and 48 kHz. It can also encode Dolby Digital-3 (AC-3), MPEG-4 Advanced Audio Codec – Low Complexity (AAC-LC), and High-Efficiency Advanced Audio Coding (HE-AAC).

As analog audio input, it supports one Stereo (two channels) with a frequency range from 20Hz to 20KHz. As digital audio input, it supports Audio Engineering Society-European Broadcasting Union (AES-EBU) and Serial digital interface/High-Definition Multimedia Interface (SDI/HDMI) (H.264 only).
As output, both ASI and IP ports deliver MPEG-2 with a bit rate from 3.4 to 19.39 Mbps.

The \textit{Futura II}'s most relevant features are 800 milliseconds latency, user-selectable resolution and bit rate, and two encoding modes, Constant Bit Rate (CBR) and Variable Bitrate (VBR).

\subsection{SOFTWARE: Twolame/Toolame/Lame}

The \textit{LAME}~\cite{lame} is a high-quality MPEG Layer III audio encoder, licensed under the Lesser General Public License (LGPL)~\cite{lgpl} and considered the best MP3 encoding software at mid-high and variable bitrates.\\
As attributes, this software delivers better quality compared to all other encoders at most bitrates, better quality and speed compared to ISO reference software, and three different encoding modes (CBR, VBR, and Average Bit Rate (ABR)). This encoder is also free format and compilable as a shared library (on Linux/Unix) and Dynamic link library (DLL)~\cite{dll} (on Windows).

Based on portions of \textit{LAME} and ISO dist10 code, the \textit{TooLAME}~\cite{toolame} was developed as a free software MPEG-1 Layer II audio encoder, written primarily by Mike Cheng. \textit{TooLAME} became well-known and widely used for its particularly high audio quality, despite existing many MP2 encoders.

Despite being unmaintained since 2003, this software was directly succeeded by the \textit{TwoLAME}~\cite{twolame} code fork, which is the focus of this work.
\textit{TwoLAME} is an optimized MPEG Layer 2 audio encoder, based on \textit{TooLAME}, with its latest version released in 2019.
Table~\ref{tab:twolame} describes the additional features not provided in the original \textit{TooLame}.

\vspace{0.5cm}

\begin{table}[H]
    \centering
    \begin{tabular}{|c|}
        \hline
        \textbf{Features} \\
        \hline
         Static and shared library (\textit{libtwolame}) \\
         \hline
         Fully thread-safe\\
         \hline
         API similar to \textit{LAME}'s (easy porting) \\
         \hline
         Front-end supports a wider range of input files \\
         \hline
         \textit{automake}/\textit{libtool}/\textit{pkgconfig} based build system \\
         \hline
         Written in Standard C (ISO C99 compliant)\\
         \hline
    \end{tabular}
    \caption{Table of \textit{TwoLame} features.}
    \label{tab:twolame}
\end{table}

