%In this section, one should summarize the work: the problem, the current work, the proposed solution, the results obtained, and why they improve the state-of-art. Repeating what is written in the introduction is not a problem and is desirable, but one should try to use different and complementary explanations to help the user. 

%Background
This work starts by describing the MPEG working group, who creates standards for
media coding, established by the ISO~\cite{iso} and IEC~\cite{iec}. Since its
formation in 1988, the MPEG working group has made an indelible mark on the
transition from analog to digital video. In particular, an increasing number of
broadcasting applications are based on MPEG technology, such as DSS~\cite{dss},
DAB~\cite{dab}, and DVB~\cite{dvb}.

%MPEG, 1 e 2, audio
The MPEG standard includes two variations, MPEG-1 and MPEG-2, with both covering
audio and video compression. Today, MPEG-1 Audio is the most widely compatible
lossy audio format in the world. The MPEG standard also differentiates three
coding schemes, called Layer I, Layer II, and Layer III.  The first two layers
are the most relevant ones. Layer I has the lowest complexity, while Layer II
requires a more complex encoder and decoder, being directed towards one-to-many
applications.

%Motivation
When developing SoCs, the developer faces tough performance-cost trade-offs and
must attempt to save on any of the components. For an audio encoder, one option
is buying an encoder chip, which increases area/volume, weight, and power
consumption. Example chips are the \textit{CX23415 MPEG-2 Codec}~\cite{cx23415},
or the \textit{Futura II ASI+IP}~\cite{futura}). Another option is using an
encoder software such as TwoLAME~\cite{twolame}. The software requires a Central
Processing Unit (CPU), either one already available, or an additional CPU IP
chip or IP core. The software also needs to be ported to the CPU. The last
option is licensing an IP Core. This reduces reduce area/volume, weight, and
power consumption, allowing the user to develop a top-notch system, beating the
competition. However, this research only found one commercial MPEG-1/2 Layer II
IP core in the market, the \textit{CWda74}, later
re-branded \textit{IPB-MPEG-SE}, which used fixed-point computations.

%RISC-V-based MPEG1/2 Layer II Encoder
This work proposes a floating-point MPEG-1/2 Layer II Audio encoder, beatint the
CWda74 in terms of accuracy, but inevitably consuming more silicon area and
burning more power. This provides a new and interesting trade-off in a market
dominated by a singe product.
 
This report analyses the \textit{IObundle, Lda}'s~\cite{iobundle} IOb-SoC, a
System-on-Chip template comprising an open-source RISC-V processor. The IOb-SoC
components, deliverables, FPGA resources, and repository are also analyzed.  In
addition, this work studies the ISO/IEC 11172 standard, an international
standard under the title \textit{Information technology - Coding of moving
pictures and associated audio for digital storage media at up to about 1,5
Mbit/s}. The standard includes an audio part, which specifies the coded
representation of high-quality audio for storage media, as well as the working
principle of a generic audio encoder and a Psychoacoustic encoder. This work
also analyses the porting of the \textit{TwoLAME} software to IOb-SoC's CPU
(PicoRV32), which may require an accelerator to allow execution in real-time.

%Work Plan
Finally, this work plans the IP development, which will be carried on a fork of
the \textit{iob-mp2-e} \textit{GitHub} repository,
including \textit{iob-cache}, \textit{iob-picorv32}, \textit{twolame},
and \textit{iob-uart} as git submodules. Targeting October 31 as the submission
deadline, the work phases are briefly described and scheduled, with the support
of a Gantt chart.

%Another vital part of this section is to explain the future work that can be done as a follow-up to this work.
