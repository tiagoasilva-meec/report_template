%In this section, one should summarize the work: the problem, the current work, the proposed solution, the results obtained, and why they improve the state of the art. Repeating what is written in the introduction is not a problem and is desirable, but one should try to use different and complementary explanations to help the user. 

%Background
Established by the ISO~\cite{iso} and IEC~\cite{iec}, the MPEG working group creates standards for media coding. Since its formation in 1988, the MPEG working group has made an indelible mark on the transition from analog to digital video. In particular, an increasing number of broadcasting applications are based on MPEG technology, such as DSS~\cite{dss}, DAB~\cite{dab}, and DVB~\cite{dvb}.\\
The MPEG standard includes two variations, MPEG-1 and MPEG-2, with both covering audio and video compression. Today, MPEG-1 Audio is the most widely compatible lossy audio format in the world. The MPEG standard also differentiates three coding schemes, called Layer I, Layer II, and Layer III.  The first two layers are the most relevant ones. Layer I has the lowest complexity, while Layer II requires a more complex encoder and decoder, being directed towards one-to-many applications.

%Motivation
When developing a SoC, the developer faces tough performance-cost trade-offs and must attempt to save on any of the components. For an audio encoder, one option is buying an encoder chip, which increases area/volume, weight, and power consumption. Example chips are the \textit{CX23415 MPEG-2 Codec}~\cite{cx23415}, or the \textit{Futura II ASI+IP}~\cite{futura}). Another option is to utilize a software encoder, which relies on a CPU. Such software encoders are commonplace in commercial embedded processors, but it is important to consider the cost and effort associated with porting the software. The last option is licensing an IP Core. This reduces area/volume, weight, and power consumption, allowing the user to develop a top-notch system and beat the competition. However, it is difficult to find commercial MPEG-1/2 Layer II IP cores in the market, with one being the \textit{CWda74}, later re-branded \textit{IPB-MPEG-SE}, which uses fixed-point computations.\\
Therefore, this work represents a pioneering effort in presenting a RISC-V implementation.
The RISC-V architecture offers advantages such as open specifications, open-source hardware designs, and energy efficiency through hardware accelerators. Customization for encoding needs enhances resource utilization, and offloading tasks to accelerators improve overall system performance.

%SW
Regarding the software architecture, the \textbf{VEXRISCV} CPU was first added to IOb-MP2-E due to its floating-point capabilities. Then, four audio test files were generated with \textit{Audacity} software and used to verify the quality of the original \textit{TwoLAME} software. 
With the IOb-MP2-E already set up, the firmware was configured to allow execution of \textit{TwoLAME} encoding algorithm, using all the necessary functions from the \textit{libtwolame} library. 
Some software optimizations were done after porting \textit{TwoLAME} to IOb-MP2-E, concerning mathematical functions, because of their computational complexity, and memory allocation. This involved substituting the mathematical operations by tables with pre-calculated values and removing several dynamic memory operations, anticipating the execution in FPGA. 
Lastly, the \textbf{TIMER} peripheral was added to IOb-MP2-E, allowing the \textbf{profiling} of \textit{TwoLAME} in FPGA. This process involved three stages, with the final stage representing the lowest level, exercised in \textit{twolame\_psycho\_3} function.
It was proved that \textit{psycho\_3\_threshold} occupies the biggest part of the program execution, corresponding to between 37\% and 64\% of \textit{TwoLAME} total execution time, depending on the input file.

%HW
Regarding the hardware architecture, the \textit{Versat} was first added to IOb-MP2-E to allow hardware acceleration of \textit{psycho\_3\_threshold} function.
Knowing the resources available in \textit{Versat}, identified as Functional Units and Operators, two control and data paths were drawn to enhance developing efficient hardware designs.
Two hardware accelerators were developed, configured, and fully tested, namely \textit{spectrum\_search} and \textit{masking\_threshold}, with each accelerator covering part of the \textit{psycho\_3\_threshold} function.

\subsection{Achievements}

The \textit{TwoLAME}'s library was successfully ported to the IOb-MP2-E with floating-point precision, thanks to the \textit{VEXRISC} CPU.
Some software optimizations were done to reduce unnecessary operations, related to mathematical functions and memory allocation.
The \textit{profiling} was done in FPGA to determine which functions were taking most of the execution time. The \textit{psycho\_3\_threshold} was marked as the function to be accelerated using \textit{Versat}.
Two hardware accelerators were developed, each corresponding to a different part of the function.
The FPGA implementation results demonstrate the impact of using \textit{Versat} in terms of resource utilization within the FPGA. While \textit{Versat} increases certain resource utilization metrics, such as Total LUTs, Logic LUTs, and Flip-Flops, it does not affect others like DSP Blocks, RAMB36, RAMB18, and URAM.
The execution time results show a significant reduction in encoding time when using the \textit{spectrum\_search} accelerator, for all input files. The speedup achieved with this accelerator is very close to the desired speedup, demonstrating its effectiveness in improving the encoding algorithm.
The required speedup indicates that the acceleration of \textit{psycho\_3\_threshold} function is not enough to meet the real-time requirements. However, the IOB-MP2-E has the potential to be a top-notch system in the future, as the achieved performance for the worst-case scenario shows that an implementation at 240MHz would meet the goal.

\subsection{Future work}
%Another vital part of this section is to explain the future work that can be done as a follow-up to this work.

The results indicate that the proposed approach is moving in the right direction. With additional hardware accelerators, real-time audio encoding with MPEG-1/2 Layer II can likely be achieved. However, this requires running different accelerators in the same program execution, something that was not possible at the moment of this work but is already being developed by others. Moreover, custom instructions can be created to extend \textit{VEXRISCV}, attempting to accelerate \textit{TwoLAME} algorithm horizontally. \\
Once the real-time requirements are met, this work can be compared with the \textit{CWda74} IP Core.

These findings contribute to the ongoing efforts to provide more efficient and competitive solutions for audio encoding in MPEG-1/2 Layer II format, with the potential for real-world applications in broadcasting and multimedia.