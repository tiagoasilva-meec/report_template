%The Abstract is a summary of the work written for specialists. Its ideal size is one-third of a page, and it should quickly explain the problem, the current work, the proposed solution, the results obtained, and why they improve the state-of-art.

%MAX 250 words

An increasing number of broadcasting applications are based on the MPEG-1/2
Audio Layer II technology, such as Digital Satellite System (DSS)~\cite{dss},
Digital Audio Broadcast (DAB)~\cite{dab}, and Digital Video Broadcast
(DVB)~\cite{dvb}.

Systems requiring MPEG-1/2 Audio Layer II encoding among other functionalities
are in strong demand but the competion is fierce, and one is forced to try to
save on any component. Concerning the audio encoder the options are as
follows. One option is buying an encoder chip, which increases area/volume,
weight, and power consumption. Another option is using an encoder software, but
this requires a Central Processing Unit (CPU), either one already available, or
an additional CPU chip or IP core. The third option is to license an
intellectual property (IP) Core for MPEG-1/2 Layer II Audio encoding, allowing
the user to develop a top-notch system. However, there is only one IP core in
the market, the \textit{CWda74}, later re-branded \textit{IPB-MPEG-SE}, which
uses fixed-point calculations. Hence, there is much room for improvement and
innovation.

This work proposes a new MPEG-1/2 Layer II Audio encoding IP, which will be
developed using \textit{IObundle, Lda}'s IOb-SoC, a System-on-Chip template
comprising an open-source RISC-V processor. Compared to the CWda74, the new IP
will feature floating-point calculations for best accuracy. The TwoLAME GitHub
repository, an optimized MPEG Audio Layer II encoding software, will provide the
algorithm.  A hardware accelerator may also be developed and integrated with the
IOb-Soc (as a peripheral), allowing system execution in real-time.
