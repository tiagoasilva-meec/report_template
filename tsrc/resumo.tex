\hspace{0.4cm} O Grupo de Especialistas em Imagem em Movimento (MPEG) foi crucial na transição do áudio analógico para o digital. O MPEG suporta a Transmissão de Áudio Digital (DAB), com o MPEG-1 Audio a ser um formato de áudio globalmente popular. O MPEG inclui esquemas de codificação como a Camada I para baixa complexidade e a Camada II para aplicações de um-para-muitos.
No desenvolvimento de Sistemas num Chip (SoC), equilibrar o desempenho e o custo é desafiador. Esta tese apresenta um codificador de Áudio MPEG-1/2 de Camada II de vírgula flutuante, oferecendo alta precisão num mercado com recursos de Propriedade Intelectual (IP) limitados.
A arquitetura de software apresenta a Unidade Central de Processamento (CPU) \textit{VexRiscv} num cenário IOb-SoC, utilizando o algoritmo \textit{TwoLAME}. São aplicadas otimizações de software. A análise de desempenho com o periférico \textit{TIMER} destaca o impacto da função \textit{psycho\_3\_threshold}, representando 37\% a 64\% do tempo de execução.
No lado de hardware, o \textit{Versat} é integrado no IOb-SoC para acelerar a função \textit{psycho\_3\_threshold}, implicando a expansão da Advanced eXtensible Interface (AXI). Dois aceleradores, \textit{spectrum\_search} e \textit{masking\_threshold}, abordam partes da função \textit{psycho\_3\_threshold}. A implementação em campo de matriz de portas programável (FPGA) mostra um aumento nas Tabelas de Consulta Total (LUTs), Flip-Flops e Blocos de Processador de Sinais Digitais (DSP).
O tempo de execução é significativamente reduzido com o acelerador \textit{spectrum\_search}.
Comparando com os requisitos de codificação em tempo real, o primeiro ficheiro com menos frames fica ligeiramente aquém, enquanto os outros, especialmente o cenário de pior caso, apresentam resultados promissores.