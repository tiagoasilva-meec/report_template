
%In this section, one should explain in detail the experiments designed to show the applicability or superiority of the proposed approach and the results obtained. Use tables or figures to facilitate a rapid apprehension of the results.

%For example, if the goal is to design a new platform to accelerate a software application, the execution times of the application running on the existing and new platforms should be presented.

%The resources used should be detailed: the frequency of operation, memory size, power, energy consumption, and communication bandwidth are examples. For FPGA implementations, for example, the number of block RAMs, LUTs, and DSPs should be presented in a Table for the configurations studied.

This chapter addresses the experimental results for the \textit{TwoLAME} encoder running on IOb-Soc, with and without \textit{Versat}.
It starts by presenting the implementation results in FPGA. Then, the execution times are analyzed. In the end, the results obtained are compared with the real-time encoding requirements.

\subsection{FPGA implementation}
This work consists in accelerating the \textit{psycho\_3\_threshold} function in IOb-SoC using \textit{Versat}. This requires more hardware, which indeed affects the resource consumption in FPGA, compared to the implementation without hardware acceleration. \\
All the work was tested using \textit{Xilinx XCKU040-1FBVA676 FPGA}. This device features 1,760K logic cells, making it capable of handling complex digital logic designs. It also boasts 7,560Kb of on-chip memory, which facilitates efficient data storage and manipulation. In addition, the FPGA offers 2,520 DSP slices, enabling high-performance signal processing tasks. \\ 

Table \ref{implementation} shows the implementation results of \textit{TwoLAME} running in IOb-SoC without \textit{Versat} and with \textit{Versat}, first using \textit{spectrum\_search} accelerator, then using \textit{masking\_threshold} accelerator.

\vspace{1cm}

\begin{table}[H]
    \centering
    \begin{tabular}{c|c|c|c|c|c|}
    \cline{2-4}
    \multicolumn{1}{c|}{}  &  \multirow{2}{*}{\parbox{2.5cm}{\centering \textbf{without \textit{Versat}}}} & \multicolumn{2}{c|}{\textbf{with \textit{Versat}}} \\
    \cline{1-1} \cline{3-4}
    \multicolumn{1}{|c|}{\textbf{Metric}} & & \textit{spectrum\_search} & \textit{masking\_threshold} \\
    \hline
    \multicolumn{1}{|c|}{Total LUTs}  & 24489 &  26198 & 26590\\ 
    \hline
    \multicolumn{1}{|c|}{Logic LUTs}  & 19818 &  21346 & 21738\\ 
    \hline
    \multicolumn{1}{|c|}{LUTRAMs}  & 4288 & 4464 & 4464\\ 
    \hline
    \multicolumn{1}{|c|}{SRLs}  & 383 & 388 & 388\\ 
    \hline
    \multicolumn{1}{|c|}{Flip-Flops}  & 24888 & 27347 & 28027 \\ 
    \hline
    \multicolumn{1}{|c|}{RAMB36}  & 157 &  157 & 157 \\ 
    \hline
    \multicolumn{1}{|c|}{RAMB18}  & 5 &  5 & 5\\ 
    \hline
    \multicolumn{1}{|c|}{URAM}  & 0 &  0 & 0\\ 
    \hline
    \multicolumn{1}{|c|}{DSP Blocks}  & 10 & 10 & 10\\ 
    \hline
    \end{tabular}
    \caption{FPGA implementation results in IOb-SoC with and without \textit{Versat}.}
    \label{implementation}
\end{table}


The previous information offers valuable insights into the influence of \textit{Versat} on various hardware utilization metrics, highlighting the potential advantages and trade-offs of its inclusion in the IOb-SoC. A brief description of each metric is presented below.

\begin{itemize}
    
\item \textbf{Total LUTs} represent the overall utilization of combinational logic elements within an FPGA. LUTs are fundamental components for implementing digital logic circuits.

\item \textbf{Logic LUTs} is a subset of LUTs used specifically for implementing logic operations in an FPGA. These LUTs store and compute logical functions and decisions.

\item \textbf{LUTRAMs} are a specialized type of memory in an FPGA that allows data storage and retrieval within a LUT structure. They combine the functions of both LUTs and small memory units.

\item \textbf{Shift Register LUTs} (SRLs) are LUTs configured for implementing shift register operations, where data is shifted bit by bit within a sequence. They are often used for tasks involving data serialization and deserialization.

\item \textbf{Flip-Flops} (FFs) are sequential storage elements within an FPGA. They are used to store and transfer data over time and are essential for implementing memory elements, registers, and clocked logic circuits.

\item \textbf{RAMB36} represents dedicated memory blocks within the FPGA that can store 36,000 bits of data. These blocks are typically used for larger data storage and retrieval tasks.

\item \textbf{RAMB18} represents dedicated memory blocks within the FPGA that can store 18,000 bits of data. These are suitable for tasks requiring smaller memory storage.

\item \textbf{URAM} (UltraRAM) is a type of high-capacity memory resource within some FPGAs. It offers significantly larger memory storage compared to regular RAM blocks and is used for high-performance memory-intensive applications.

\item \textbf{DSP Blocks} (Digital Signal Processor Blocks) are specialized resources in an FPGA designed for accelerating digital signal processing tasks. They include dedicated hardware for tasks like multiplication, accumulation, and complex arithmetic operations.

\end{itemize}

Focusing on the measurement without \textit{Versat} and with \textit{Versat} using \textit{spectrum\_search} accelerator, the results show a modest increase from 24,489 Total LUTs without \textit{Versat} to 26,198 Total LUTs with \textit{Versat}. This suggests that \textit{Versat} consumes additional LUT resources within the FPGA, as expected.
The inclusion of \textit{Versat} increases from 19,818 Logic LUTs to 21,346, affirming its impact on logic operations within the system. It also exhibits a change in LUTRAMs from 4,288 to 4,464, indicating the utilization of dedicated memory resources within the FPGA.
The small difference between the two scenarios, from 383 SRLs to 388 SRLs, suggests that \textit{Versat} minimally affects this particular resource. On the opposite, the significant jump in Flip-Flops from 24,888 without \textit{Versat} to 27,347 with it is notable, indicating a requirement increase for sequential storage elements within the FPGA.
Interestingly, there is no difference in the number of RAMB36, RAMB18, URAM, and DSP Blocks resources, suggesting that \textit{Versat} does not utilize this specific resource.

The measurement results with \textit{Versat} using \textit{masking\_threshold} accelerator are similar to the \textit{spectrum\_search} accelerator ones. More precisely, the \textit{masking\_threshold} accelerator requires a few more Total LUTs, Logic LUTs, and Flip-Flops, compared to the other accelerator.

% This information can guide system architects and designers in making informed decisions about hardware resource allocation within the FPGA.

\subsection{Execution results}

Table \ref{time} shows the execution time of \textit{TwoLAME} running in IOb-SoC without \textit{Versat} and with \textit{Versat} using \textit{spectrum\_search} accelerator, for all input files.

\vspace{1cm}

\begin{table}[H]
    \centering
    \begin{tabular}{|c|c|c|c|c|c|}
    \cline{3-6}
    \multicolumn{2}{c|}{}  & \multicolumn{4}{c|}{\textbf{Input files}} \\
    \cline{3-6}
    \multicolumn{2}{c|}{} & \textit{short.wav} & \textit{long.wav} & \textit{noise.wav} & \textit{vivaldi.wav} \\
    \hline
   \multirow{2}{*}{\parbox{2.5cm}{\centering \textbf{without} \\ \textbf{\textit{Versat}}}}  & \multicolumn{1}{c|}{\textit{psycho\_3\_threshold}}  & 536 & 13365 & 11381 & 14694 \\ 
    \cline{2-6}
    & \multicolumn{1}{|c|}{\textit{TwoLAME} total}  & 1509 & 22672 & 18520 & 24325\\ 
    \hline
    \multirow{2}{*}{\parbox{3.2cm}{\centering with \\ \textbf{\textit{spectrum\_search}}}}  & \multicolumn{1}{c|}{\textit{psycho\_3\_threshold}}  & 20 & 350 & 289 & 395 \\ 
    \cline{2-6}
    & \multicolumn{1}{|c|}{\textit{TwoLAME} total}  & 974 & 9391 & 7209 & 9729 \\ 
    \hline
    \end{tabular}
    \caption{Execution time for all input files with and without \textit{Versat} [ms].}
    \label{time}
\end{table}

The previous information shows that the \textit{spectrum\_search} \textit{Versat} accelerator can induce a substantial reduction in execution time. Specifically, the execution times for \textit{short.wav}, \textit{long.wav}, \textit{noise.wav} and \textit{vivaldi.wav} are reduced by approximately 96.26\%, 94.6\%, 95.34\% and 94.98\%, respectively.
Considering that the \textit{spectrum\_search} accelerator executes part of the original function and not the whole function, these results are extremely positive. Moreover, with a reduction of 94-96\% for all input files, the \textit{masking\_threshold} accelerator becomes unnecessary. That is because \textit{Versat} does not allow execution of different accelerators in the same run, but rather the execution of a single accelerator, with different configurations (the case of this work) or not.
Therefore, the \textit{masking\_threshold} accelerator was not tested in FPGA, since it would hurt the execution times (mainly due to the switching between software and hardware). After being developed, this accelerator was only tested for correctness using emulation on a PC.


\subsection{Real-time requirements}

%The previous table shows interesting information. First, it is noticeable that several functions have an insignificant execution time, compared to others. Second, it is clear that \textit{psycho\_3 function 11}, which is \textit{psycho\_3\_threshold}, occupies the biggest part of the program execution. For \textit{short.wav}, \textit{long.wav}, \textit{noise.wav} and \textit{vivaldi.wav}, this function corresponds to 37\%, 61\%, 64\% and 63\% of \textit{TwoLAME} execution time, respectively.

After measuring the execution time of the original and the hardware-accelerated implementation in IOb-SoC, it is possible to calculate the speedup achieved for each input file, using the formula presented below.

\begin{equation}
    \textit{Speedup achieved} = \frac{\textit{Execution time without acceleration}}{\textit{Execution time with acceleration}} 
\end{equation}

Additionally, the \textit{Amdahl's Law} can be applied to understand the limitations of the hardware acceleration, for the \textit{psycho\_3\_threshold} function.

\begin{equation}
    \textit{Speedup desired} = \frac{1}{(1 - p) + \frac{p}{s}} 
\end{equation}

In the formula above, $s$ is the speedup of the part of the task that benefits from improved system resources, i.e. is the speedup achieved on the parallelizable portion; $p$ is the proportion of execution time that the part benefiting from improved resources originally occupied, i.e. is the proportion of the program that can be parallelized.

In the ideal case where the parallelizable part ($p$) disappears or, in other words, where the speedup ($s$) becomes very high, the value of $s$ approaches infinity, and the formula simplifies accordingly.

\begin{equation}
    \textit{Speedup desired} \approx \frac{1}{1 - p}
\end{equation}

%in meeting the specified real-time constraints.
A more interesting point is to compare the achieved speedup with the desired speedup based on the real-time requirements, for each input file. 

This involves a formula to estimate how long it would take to encode the audio file in real time, which is presented below.

\begin{equation}
    \textit{Real-time} = \frac{\textit{number of frames} \times \textit{number of samples}}{\textit{sampling frequency}} 
\end{equation}

In this case, the real-time is calculated from the number of frames, the number of samples, and the sampling frequency.
The \textit{number of samples} refers to the total number of audio samples in the audio file, which are individual data points that represent the amplitude of the audio signal at a particular point in time. This value is multiplied by the \textit{number of frames} since each frame typically consists of a fixed number of audio samples to process audio data efficiently. The result of the product is then divided by the \textit{sampling frequency}, a fundamental parameter of digital audio that specifies how many samples are taken per second to represent the analog audio signal.

Knowing the real-time, the speedup required can be determined by the following formula.

\begin{equation}
    \textit{Speedup required} = \frac{\textit{Execution time without acceleration}}{\textit{Real-time}} 
\end{equation}

Table \ref{speedup} presents the outcomes of the previously discussed calculations.

\vspace{1cm}

\begin{table}[H]
    \centering
    \begin{tabular}{|c|c|c|c|c|}
    \cline{2-5}
    \multicolumn{1}{c|}{}  & \multicolumn{4}{c|}{\textbf{Input file}} \\
    \cline{1-5}
    \multicolumn{1}{|c|}{\textbf{Speedup}} & \textit{short.wav} & \textit{long.wav} & \textit{noise.wav} & \textit{vivaldi.wav} \\
    \hline
    \multirow{2}{*}{\parbox{3.2cm}{\centering achieved with \\ \textbf{\textit{spectrum\_search}}}} & \multirow{2}{*}{\centering 1.549} & \multirow{2}{*}{\centering 2.414} & \multirow{2}{*}{\centering 2.569} & \multirow{2}{*}{\centering 2.5}  \\ 
    & & & &  \\ 
    \hline
    \multicolumn{1}{|c|}{desired}  & 1.551 & 2.436 & 2.594  & 2.526 \\ 
    \hline
    \multicolumn{1}{|c|}{required}  & 5.251 & 2.9321 & 6.219 & 5.748 \\ 
    \hline
    \end{tabular}
    \caption{Calculation results for all input files.}
    \label{speedup}
\end{table}

%real-time 287.3469   7732.2448   2977.9591   4231.8367

In more detail, the achieved speedup is first calculated based on the \textit{TwoLAME total} execution times from table \ref{time}, showing the performance gain compared to the original \textit{TwoLAME} software in IOb-SoC.\\
Then, the desired speedup is calculated taking into account the proportion of the \textit{TwoLAME} that can be parallelized, which is, in practice, the result of the division between \textit{psycho\_3\_threshold} and \textit{TwoLAME total} execution times without \textit{Versat}, also from table \ref{time}. \\
Lastly, the required speedup is calculated with a \textit{sampling frequency} of $44100Hz$ and a \textit{number of samples} of $1152$, for all input files. It also considers the \textit{TwoLAME total} execution times without \textit{Versat}. The remaining variable, \textit{number of frames}, is different for each input file. In concrete, \textit{short.wav}, \textit{long.wav}, \textit{noise.wav} and \textit{vivaldi.wav} contain $11$, $296$, $114$ and $162$ frames, respectively.

Focusing on the results, it is noticeable that the \textit{spectrum\_search} \textit{Versat} accelerator provides a speedup very close to the desired one (which is the maximum possible speedup achievable by accelerating \textit{psycho\_3\_threshold} function).
However, the required speedup indicates that the acceleration of \textit{psycho\_3\_threshold} is not enough to meet the real-time requirements. 
In particular, the first two input files encoding is approximately at 3.39 and 1.21 speedup from meeting the real-time requirements, respectively. In the first case, the input file has a small number of frames, which doesn't potentiate \textit{Versat} acceleration and, consequently, the achieved speedup stays far from the required one.
In the second case, the input file has a significant number of frames but the audio itself is simple, which implies a higher demand of \textit{psycho\_3\_threshold} function (executed by \textit{Versat}) compared to the other \textit{TwoLAME} functions, and so the achieved speedup gets close to the required one. \\
The last two input files' encoding is approximately at 2.42 and 2.30 speedup from meeting the real-time requirements. These results give some guarantees about achieving the goal of this work since the third input file has a considerable number of frames and represents the worst-case scenario for real-time encoding (being a stereo white noise audio file). The fourth input file has a normal audio complexity since it is part of an existing soundtrack, and so the achieved and the required speedups are also not far away.

%In the second case, the input file has a significant number of frames, but the audio is simple so it does not take advantage of \textit{Versat} acceleration, due to the low demand of \textit{psycho\_3\_threshold} function (in the original software).
