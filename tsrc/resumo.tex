\hspace{0.4cm} Este trabalho apresenta um codificador de Áudio MPEG-1/2 de Camada II de vírgula flutuante, oferecendo alta precisão num mercado com recursos de Propriedade Intelectual (IP) limitados.
A arquitetura de software apresenta a Unidade Central de Processamento (CPU) \textit{VexRiscv} num cenário IOb-SoC, utilizando o algoritmo \textit{TwoLAME}. São aplicadas otimizações de software. A análise de desempenho com o periférico \textit{TIMER} destaca o impacto da função \textit{psycho\_3\_threshold}, representando 37\% a 64\% do tempo de execução.
No lado de hardware, o \textit{Versat} é integrado no IOb-SoC para acelerar a função \textit{psycho\_3\_threshold}, implicando a expansão da Advanced eXtensible Interface (AXI). Dois aceleradores, \textit{spectrum\_search} e \textit{masking\_threshold}, abordam partes da função \textit{psycho\_3\_threshold}. A implementação em campo de matriz de portas programável (FPGA) mostra um aumento nas Tabelas de Consulta Total (LUTs), Flip-Flops e Blocos de Processador de Sinais Digitais (DSP).
O tempo de execução é significativamente reduzido com o acelerador \textit{spectrum\_search}.
Comparando o aumento de velocidade alcançado com o aumento de velocidade necessário para a codificação em tempo real, alguns ficheiros de entrada aproximam-se de cumprir os requisitos. Apesar de o primeiro ficheiro, com um número reduzido de frames, ficar áquem dos requisitos, os restantes ficheiros, especialmente o que representa o cenário de pior caso, apresentam resultados promissores.
