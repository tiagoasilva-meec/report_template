%In this section, one should explain in detail the experiments designed to show the applicability or superiority of the proposed approach and the results obtained. Use tables or figures to facilitate a rapid apprehension of the results.

%For example, if the goal is to design a new platform to accelerate a software application, the execution times of the application running on the existing and new platforms should be presented.

%The resources used should be detailed: the frequency of operation, memory size, power, energy consumption, and communication bandwidth are examples. For FPGA implementations, for example, the number of block RAMs, LUTs, and DSPs should be presented in a Table for the configurations studied.

This chapter addresses the experimental results for the IOb-MP2-E running on FPGA, with and without hardware accelerators.
It starts by presenting the results of \textit{TwoLAME} profiling. Then, it presents the implementation results in FPGA. Afterward, the execution times are analyzed. In the end, the results obtained are compared with the real-time encoding requirements.

\subsection{Profiling}

The results of the first stage of \textit{TwoLAME}'s profiling are shown in the table below. The values below 1 ms are represented by a hyphen ('-').

\begin{table}[H]
    \centering
    \begin{tabular}{c|c|c|c|c|}
    \cline{2-5}
    \multicolumn{1}{c|}{}  & \multicolumn{4}{c|}{\textbf{Input file}} \\
    \cline{2-5}
     & short.wav & long.wav & noise.wav & vivaldi.wav \\
    \hline
    \multicolumn{1}{|c|}{twolame\_init()}  & 3 & 2 & 3 & - \\ 
    \hline
    \multicolumn{1}{|c|}{wave\_init()}  & - & - & - & - \\ 
    \hline
    \multicolumn{1}{|c|}{twolame\_set\_num\_channels()}   & - & - & - & - \\ 
    \hline
    \multicolumn{1}{|c|}{twolame\_set\_in\_samplerate()}   & - & - & - & - \\ 
    \hline
    \multicolumn{1}{|c|}{twolame\_set\_bitrate()}   & - & - & - & - \\ 
    \hline
    \multicolumn{1}{|c|}{twolame\_init\_params()}   & 11 & 11 & 11 & 11 \\ 
    \hline
    \multicolumn{1}{|c|}{wave\_get\_samples()}   & 4 & 126 & 89 & 135 \\ 
    \hline
    \multicolumn{1}{|c|}{twolame\_encode\_buffer\_interleaved()}   & 1510 & 24345 & 19820 & 26010 \\ 
    \hline
    \multicolumn{1}{|c|}{memcpy()}  & 2 & 59 & 19 & 32 \\ 
    \hline
    \multicolumn{1}{|c|}{twolame\_encode\_flush()}   & 79 & 80 & 185 & 155 \\ 
    \hline
    \multicolumn{1}{|c|}{memcpy()}  & - & - & - & - \\ 
    \hline
    \multicolumn{1}{|c|}{twolame\_close()}   & - & - & - & 1 \\ 
    \hline
    \multicolumn{1}{|c|}{\textbf{\textit{TwoLAME} total}}  & 1614 & 24719 & 20210 & 26467 \\ 
    \hline
    \end{tabular}
    \caption{Execution time for all input files (first phase of \textit{profiling}) [ms].}
    \label{profiling1}
\end{table}

The previous information shows that \textit{twolame\_encode\_buffer\_interleaved} occupies most of the execution time, which is expected as it is responsible for the encoding, calling many other \textit{TwoLAME} functions. Nonetheless, this information is not enough, mainly because developing hardware to accelerate the whole function would be an extremely difficult task.
Therefore, the results of the second stage of \textit{TwoLAME}'s profiling are shown in the table below.

\begin{table}[H]
    \centering
    \begin{tabular}{|c|c|c|c|c|}
    \cline{2-5}
    \multicolumn{1}{c|}{}  & \multicolumn{4}{c|}{\textbf{Input file}} \\
    \cline{2-5}
    \multicolumn{1}{c|}{} & \textit{short.wav} & \textit{long.wav} & \textit{noise.wav} & \textit{vivaldi.wav} \\
    \hline
    \multicolumn{1}{|c|}{\textit{scale\_and\_mix\_samples}}  & - & 9 & 5 & 3 \\ 
   \hline
    \multicolumn{1}{|c|}{\textit{twolame\_buffer\_sstell}} & 3 & 139 & 52 & 78 \\ 
    \hline
     \multicolumn{1}{|c|}{\textit{twolame\_available\_bits}} & - & 16 & 7 & 8  \\ 
    \hline
     \multicolumn{1}{|c|}{\textit{twolame\_window\_filter\_subband}} & 141  & 3487 & 2590 & 3662  \\ 
    \hline
     \multicolumn{1}{|c|}{\textit{twolame\_scalefactor\_calc}} & 7 & 172 & 142 &  201  \\ 
    \hline
     \multicolumn{1}{|c|}{\textit{twolame\_find\_sf\_max}} & -  & 13 & 3 & 6   \\ 
    \hline
     \multicolumn{1}{|c|}{\textit{twolame\_scalefactor\_calc}} & -  & 5 & 6 & 4  \\ 
    \hline
      \multicolumn{1}{|c|}{\textit{twolame\_psycho\_3}} & 1377  & 19062 & 16222 &  20829 \\ 
     \hline
     \multicolumn{1}{|c|}{\textit{twolame\_sf\_transmission\_pattern}} & - & 17 & 8 &  10 \\ 
    \hline
     \multicolumn{1}{|c|}{\textit{twolame\_main\_bit\_allocation}} & 16  & 424 & 343 & 485  \\ 
    \hline
     \multicolumn{1}{|c|}{\textit{twolame\_write\_header}} & 2 & 12 & 7 &  7 \\
    \hline
     \multicolumn{1}{|c|}{\textit{buffer\_putbits}} & - & 4 & 1 & 4  \\
    \hline
     \multicolumn{1}{|c|}{\textit{twolame\_write\_bit\_alloc}} & - & 11 & 11 &  17 \\
    \hline
     \multicolumn{1}{|c|}{\textit{twolame\_write\_scalefactors}} & 1 & 22 & 20 & 27 \\
    \hline
     \multicolumn{1}{|c|}{\textit{twolame\_subband\_quantization}} & 21 & 503 & 346 & 488 \\
    \hline
     \multicolumn{1}{|c|}{\textit{twolame\_write\_samples}} & 12 & 306 & 149 & 210 \\
    \hline
     \multicolumn{1}{|c|}{\textit{buffer\_put1bit}} & - & 10 & 4 & 2 \\
    \hline
     \multicolumn{1}{|c|}{\textit{buffer\_putbits}} & - & 5 & 4 & 2 \\
    \hline
     \multicolumn{1}{|c|}{\textit{twolame\_dab\_crc\_calc}} & - & 9 & 3 &  3 \\
    \hline
     \multicolumn{1}{|c|}{\textit{buffer\_put1bit}} & - & 3 & - &  - \\
    \hline
     \multicolumn{1}{|c|}{\textit{twolame\_buffer\_sstell}} & - & 1 & - & 1  \\
    \hline
     \multicolumn{1}{|c|}{\textit{twolame\_do\_energy\_levels}} & - & 7 & 1 & 3  \\
    \hline
     \multicolumn{1}{|c|}{\textit{twolame\_crc\_writeheader}} & - & 3 & 1 &  2  \\ 
    \hline
    \multicolumn{1}{|c|}{\textbf{\textit{TwoLAME} total}}  & 1614 & 24719 & 20210 & 26467 \\ 
    \hline
    \end{tabular}
    \caption{Execution time for all input files (second phase of \textit{profiling}) [ms].}
    \label{profiling2}
\end{table}

The previous table shows that the eight block of \textit{encode\_frame()} occupies most of the execution time, which includes two nested loops (executed depending on \textit{if} conditions) and a switch case. The nested loops perform simple operations, so they are not relevant. The switch case calls a certain function depending on the switch condition. However, after inspecting \textit{glopts->psymodel} condition it is perceptible that case 3 is always selected, calling \textit{twolame\_psycho\_3}. Nevertheless, this information is still not enough and the results of the third stage of \textit{TwoLAME}'s profiling are shown in the table below.

\begin{table}[H]
    \centering
    \begin{tabular}{|c|c|c|c|c|}
    \cline{2-5}
    \multicolumn{1}{c|}{}  & \multicolumn{4}{c|}{\textbf{Input file}} \\
    \cline{2-5}
    \multicolumn{1}{c|}{} & short.wav & long.wav & noise.wav & vivaldi.wav \\
    \hline
    \multicolumn{1}{|c|}{twolame\_psycho\_3\_init}  & 635 & 642 & 635 & 635 \\ 
   \hline
    \multicolumn{1}{|c|}{\textit{mem} $\rightarrow$ \textit{fft\_buf}} & 5 & 133 & 115 & 155 \\ 
    \hline
     \multicolumn{1}{|c|}{\textit{mem} $\rightarrow$ \textit{fft\_buf}} &  6 &  117& 77 &  120 \\ 
    \hline
     \multicolumn{1}{|c|}{\textit{psycho\_3\_fft}} & 47 & 1138 & 886 &  1253  \\ 
    \hline
     \multicolumn{1}{|c|}{\textit{psycho\_3\_powerdensityspectrum}} & 40  & 1026 & 791 &  1115  \\ 
    \hline
     \multicolumn{1}{|c|}{\textit{psycho\_3\_spl}} &  7 & 170 & 148 & 204  \\ 
    \hline
      \multicolumn{1}{|c|}{\textit{psycho\_3\_tonal\_label}} &  7 & 153 & 258 & 173  \\ 
     \hline
     \multicolumn{1}{|c|}{\textit{psycho\_3\_noise\_label}} & 8 & 220 & 167 & 235  \\ 
    \hline
     \multicolumn{1}{|c|}{\textit{psycho\_3\_dump}} &  - & 4 & 2 & 4  \\ 
    \hline
     \multicolumn{1}{|c|}{\textit{psycho\_3\_decimation}} & 3 & 45 & 38 & 67   \\
    \hline
     \multicolumn{1}{|c|}{\textit{psycho\_3\_threshold}} & 611 & 15168  & 12975 &  16684 \\
    \hline
     \multicolumn{1}{|c|}{\textit{psycho\_3\_minimummasking}} & - & 23 & 14 & 29  \\ 
    \hline
     \multicolumn{1}{|c|}{\textit{psycho\_3\_smr}} & - & 9 & 6 & 13  \\ 
    \hline
    \multicolumn{1}{|c|}{\textbf{\textit{TwoLAME} total}}  & 1614 & 24719 & 20210 & 26467 \\ 
    \hline
    \end{tabular}
    \caption{Execution time for all input files (third phase of \textit{profiling}) [ms].}
    \label{profiling3}
\end{table}

The previous table shows interesting information. First, it is noticeable that several functions have an insignificant execution time, compared to \textit{TwoLAME} total execution time. Second, it is clear that \textit{psycho\_3 function 11}, which corresponds to \textit{psycho\_3\_threshold} function, occupies the biggest part of the program execution. For \textit{short.wav}, \textit{long.wav}, \textit{noise.wav} and \textit{vivaldi.wav}, this function corresponds to 37\%, 61\%, 64\% and 63\% of \textit{TwoLAME} execution time, respectively.
This motivates the hardware acceleration of \textit{psycho\_3\_threshold}, described in \textit{Hardware architecture} section.

\subsection{FPGA implementation}
This work consists of accelerating the \textit{psycho\_3\_threshold} function in IOb-MP2-E using hardware acceleration. This requires more hardware, which indeed affects the resource consumption in FPGA, compared to the implementation without hardware acceleration.
All the work was tested using \textit{KU040 Xilinx Kintex® UltraScale™ Development Kit}~\cite{bib:xcku040}. This chip is based on the 28nm \textit{UltraScale+} architecture, containing \textit{Xilinx XCKU040-1FBVA676 FPGA}, 250 MHz LVDS Oscillator (system clock), 1GB DDR4 SDRAM and USB-UART Interface. As internal resources, it allows a total of 242400 LUTs, 484800 FFs, 600 BRAMs, and 1920 DSPs.
In this work, the Xilinx board runs at 100MHz, the frequency at which IOb-MP2-E is compiled. 

Table \ref{implementation} shows the implementation results of IOb-MP2-E without \textit{Versat} and with \textit{Versat}, first using \textit{spectrum\_search} accelerator, then using \textit{masking\_threshold} accelerator.

\vspace{0.5cm}

\begin{table}[H]
    \centering
    \begin{tabular}{|c|c|c|c|c|c|}
    \hline
     \multirow{2}{*}{\parbox{2.5cm}{\centering \textbf{Metric}}}  &  \multirow{2}{*}{\parbox{2.5cm}{\centering \textbf{without \textit{Versat}}}} & \multicolumn{2}{c|}{\textbf{with \textit{Versat}}} \\
    \cline{3-4}
     & & \textit{spectrum\_search} & \textit{masking\_threshold} \\
    \hline
    \multicolumn{1}{|c|}{Total LUTs}  & 24489 &  38913 & 34810\\ 
    \hline
    \multicolumn{1}{|c|}{Logic LUTs}  & 19818 &  33359 & 29267\\ 
    \hline
    \multicolumn{1}{|c|}{LUTRAMs}  & 4288 & 5160 & 5152\\ 
    \hline
    \multicolumn{1}{|c|}{SRLs}  & 383 & 394 & 391\\ 
    \hline
    \multicolumn{1}{|c|}{Flip-Flops}  & 24888 & 37653 &  35186\\ 
    \hline
    \multicolumn{1}{|c|}{RAMB36}  & 157 &  161 &  176\\ 
    \hline
    \multicolumn{1}{|c|}{RAMB18}  & 5 &  5 & 6\\ 
    \hline
    \multicolumn{1}{|c|}{URAM}  & 0 &  0 & 0\\ 
    \hline
    \multicolumn{1}{|c|}{DSP Blocks}  & 10 & 26 & 14\\ 
    \hline
    \end{tabular}
    \caption{FPGA implementation results of IOb-MP2-E with and without \textit{Versat}.}
    \label{implementation}
\end{table}

\vspace{0.5cm}

The previous information offers valuable insights into the influence of \textit{Versat} on various hardware utilization metrics, highlighting the potential advantages and trade-offs of its inclusion in the IOb-MP2-E. A brief description of each metric is presented below.

\begin{itemize}
    
\item \textbf{Total LUTs} represent the overall utilization of combinational logic elements within an FPGA. LUTs are fundamental components for implementing digital logic circuits.

\item \textbf{Logic LUTs} is a subset of LUTs used specifically for implementing logic operations in an FPGA. These LUTs store and compute logical functions and decisions.

\item \textbf{LUTRAMs} are a specialized type of memory in an FPGA that allows data storage and retrieval within a LUT structure. They combine the functions of both LUTs and small memory units.

\item \textbf{Shift Register LUTs} (SRLs) are LUTs configured for implementing shift register operations, where data is shifted bit by bit within a sequence. They are often used for tasks involving data serialization and deserialization.

\item \textbf{Flip-Flops} (FFs) are sequential storage elements within an FPGA. They are used to store and transfer data over time and are essential for implementing memory elements, registers, and clocked logic circuits.

\item \textbf{RAMB36} represents dedicated memory blocks within the FPGA that can store 36,000 bits of data. These blocks are typically used for larger data storage and retrieval tasks.

\item \textbf{RAMB18} represents dedicated memory blocks within the FPGA that can store 18,000 bits of data. These are suitable for tasks requiring smaller memory storage.

\item \textbf{URAM} (UltraRAM) is a type of high-capacity memory resource within some FPGAs. It offers significantly larger memory storage compared to regular RAM blocks and is used for high-performance memory-intensive applications.

\item \textbf{DSP Blocks} (Digital Signal Processor Blocks) are specialized resources in an FPGA designed for accelerating digital signal processing tasks. They include dedicated hardware for tasks like multiplication, accumulation, and complex arithmetic operations.

\end{itemize}

Focusing on the measurement without \textit{Versat} and with \textit{Versat} using \textit{spectrum\_search} accelerator, the results show a significant increase from 24489 Total LUTs without \textit{Versat} to 38913 Total LUTs with \textit{Versat}. This suggests that \textit{Versat} consumes additional LUT resources within the FPGA, as expected.
The inclusion of \textit{Versat} increases from 19818 Logic LUTs to 33359, affirming its impact on logic operations within the system. It also exhibits a change in LUTRAMs from 4288 to 5160, indicating the utilization of dedicated memory resources within the FPGA.
The small difference between the two scenarios, from 383 SRLs to 394 SRLs, suggests that \textit{Versat} minimally affects this particular resource. On the opposite, the significant jump in Flip-Flops from 24888 without \textit{Versat} to 37653 with it is notable, indicating a requirement increase for sequential storage elements within the FPGA.
There is also a slight increase in RAMB36, and a big one in DSP Blocks, from 10 to 26.
Interestingly, there is no difference in the number of RAMB18 and URAM resources, suggesting that \textit{Versat} does not utilize this specific resource.

The measurement results with \textit{Versat} using \textit{masking\_threshold} accelerator are similar to the \textit{spectrum\_search} accelerator ones. More precisely, the \textit{masking\_threshold} accelerator requires a bit less resources than the previous accelerator, except for RAMB36, RAMB18, and DSP Blocks, which are all required in higher quantity.

% This information can guide system architects and designers in making informed decisions about hardware resource allocation within the FPGA.

\subsection{Execution time}

Table \ref{time} shows the execution time of IOb-MP2-E without \textit{Versat} and with \textit{Versat}, using \textit{spectrum\_search} and \textit{masking\_threshold} accelerators, for all input files.

\vspace{0.5cm}

\begin{table}[H]
    \centering
    \begin{tabular}{|c|c|c|c|c|c|}
    \cline{3-6}
    \multicolumn{2}{c|}{}  & \multicolumn{4}{c|}{\textbf{Input files}} \\
    \cline{3-6}
    \multicolumn{2}{c|}{} & \textit{short.wav} & \textit{long.wav} & \textit{noise.wav} & \textit{vivaldi.wav} \\
    \hline
   \multirow{2}{*}{\parbox{2.5cm}{\centering \textbf{without} \\ \textbf{\textit{Versat}}}}  & \multicolumn{1}{c|}{\textit{psycho\_3\_threshold}}  & 536 & 13365 & 11381 & 14694 \\ 
    \cline{2-6}
    & \multicolumn{1}{|c|}{\textit{TwoLAME} total}  & 1509 & 22672 & 18520 & 24325\\ 
    \hline
    \multirow{2}{*}{\parbox{3.2cm}{\centering with \\ \textbf{\textit{spectrum\_search}}}}  & \multicolumn{1}{c|}{\textit{psycho\_3\_threshold}}  & 20 & 350 & 289 & 395 \\ 
    \cline{2-6}
    & \multicolumn{1}{|c|}{\textit{TwoLAME} total}  & 974 & 9391 & 7209 & 9729 \\ 
    \hline
    \multirow{2}{*}{\parbox{3.2cm}{\centering with \\ \textbf{\textit{masking\_threshold}}}}  & \multicolumn{1}{c|}{\textit{psycho\_3\_threshold}}  & 555 & 13807 & 11761 & 15168 \\ 
    \cline{2-6}
    & \multicolumn{1}{|c|}{\textit{TwoLAME} total}  & 1518 & 22677 & 18637 & 24458 \\ 
    \hline
    \end{tabular}
    \caption{Execution time of IOb-MP2-E with and without \textit{Versat} for all input files [ms].}
    \label{time}
\end{table}


\vspace{0.5cm}

The previous information shows that the \textit{spectrum\_search} accelerator can induce a substantial reduction in execution time. Specifically, the \textit{psycho\_3\_threshold} execution times for \textit{short.wav}, \textit{long.wav}, \textit{noise.wav} and \textit{vivaldi.wav} are reduced by approximately 96.26\%, 94.6\%, 95.34\% and 94.98\%, respectively.
Considering that the \textit{spectrum\_search} accelerator executes part of the original function and not the whole function, these results are extremely positive. In this accelerator there is only a a main transference of data at the beginning, followed by the execution of many loop iterations, keeping the data inside the accelerator. Since only a few constants are changed in each iteration, the performance gain is big.
Moreover, with a reduction of 94-96\% for all input files, the \textit{masking\_threshold} accelerator becomes unnecessary. That is because \textit{Versat} does not allow execution of different accelerators in the same run, but rather the execution of a single accelerator, with different configurations (the case of this work) or not.
The \textit{masking\_threshold} accelerator increased the execution time by a few milliseconds for all input files, i.e. no acceleration was achieved with \textit{masking\_threshold} at all. This can be explained by the fact that the overhead of transferring data is greater than any gain in performance using the accelerator. The loop executes only 512 iterations, which is not enough to make the data transfer worthwhile.


\subsection{Real-time requirements}

%The previous table shows interesting information. First, it is noticeable that several functions have an insignificant execution time, compared to others. Second, it is clear that \textit{psycho\_3 function 11}, which is \textit{psycho\_3\_threshold}, occupies the biggest part of the program execution. For \textit{short.wav}, \textit{long.wav}, \textit{noise.wav} and \textit{vivaldi.wav}, this function corresponds to 37\%, 61\%, 64\% and 63\% of \textit{TwoLAME} execution time, respectively.

After measuring the execution time of the original and the hardware-accelerated implementation in IOb-MP2-E, it is possible to calculate the speedup achieved for each input file, using the formula presented below.

%\vspace{0.5cm}

\begin{equation}
    \textbf{\textit{Speedup achieved}} = \frac{\textit{Execution time without \textit{Versat}}}{\textit{Execution time with \textit{Versat}}} 
\end{equation}

Additionally, the \textit{Amdahl's Law}~\cite{amdahls} can be applied to understand the limitations of the hardware acceleration, for the \textit{psycho\_3\_threshold} function.

\begin{equation}
    \textit{Speedup possible} = \frac{1}{(1 - p) + \frac{p}{s}} 
\end{equation}

In the formula above, $s$ is the speedup of the part of the task that benefits from improved system resources, i.e. is the speedup achieved on the parallelizable portion; $p$ is the proportion of execution time that the part benefiting from improved resources originally occupied, i.e. is the proportion of the program that can be parallelized.

In the ideal case where the parallelizable part ($p$) disappears or, in other words, where the speedup ($s$) is very high, the value of $s$ approaches infinity, and the formula simplifies accordingly.

\begin{equation}
    \textbf{\textit{Speedup possible}} \approx \frac{1}{1 - p}
\end{equation}

%in meeting the specified real-time constraints.
A more interesting point is to compare the achieved speedup with the possible speedup based on the real-time requirements, for each input file. 
This involves a formula to estimate how long it would take to encode the audio file in real time, which is presented below.

\begin{equation}
    \textbf{\textit{Real-time}} = \frac{\textit{number of frames} \times \textit{number of samples}}{\textit{sampling frequency}} 
\end{equation}

In this case, the real-time is calculated from the number of frames, the number of samples, and the sampling frequency.
The \textit{number of samples} refers to the total number of audio samples in the audio file, which are individual data points that represent the amplitude of the audio signal at a particular point in time~\cite{twolamefrontend}. This value is multiplied by the \textit{number of frames} since each frame typically consists of a fixed number of audio samples (the case of this work) to process audio data efficiently. The result of the product is then divided by the \textit{sampling frequency}, a fundamental parameter of digital audio that specifies how many samples are taken per second to represent the analog audio signal.

The \textit{sampling frequency} is $44100Hz$ and the \textit{number of samples} is $1152$, for all input files. The remaining variable, \textit{number of frames}, is different for each input file. In concrete, \textit{short.wav}, \textit{long.wav}, \textit{noise.wav} and \textit{vivaldi.wav} contain $11$, $296$, $114$ and $162$ frames, respectively.
Table \ref{realtime} shows the real-time for all input files.

\vspace{0.5cm}

\begin{table}[H]
    \centering
    \begin{tabular}{|c|c|c|c|c|}
    \cline{2-5}
    \multicolumn{1}{c|}{}  & \multicolumn{4}{c|}{\textbf{Input file}} \\
    \cline{2-5}
    \multicolumn{1}{c|}{} & \textit{short.wav} & \textit{long.wav} & \textit{noise.wav} & \textit{vivaldi.wav} \\
    \hline
    \textbf{Real time} & 287 & 7732 & 2978 & 4232  \\ 
    \hline
    \end{tabular}
    \caption{Real time for all input files encoding [ms].}
    \label{realtime}
\end{table}

\vspace{0.3cm}

Based on the previous information, the speedup required can be determined by the following formula.

\begin{equation}
    \textit{Speedup required} = \frac{\textit{Execution time without \textit{Versat}}}{\textit{Real-time}} 
\end{equation}

Table \ref{speedup} presents the outcomes of the previously discussed calculations.

\vspace{0.5cm}

\begin{table}[H]
    \centering
    \begin{tabular}{|c|c|c|c|c|}
    \cline{2-5}
    \multicolumn{1}{c|}{}  & \multicolumn{4}{c|}{\textbf{Input file}} \\
    \cline{1-5}
    \multicolumn{1}{|c|}{\textbf{Speedup}} & \textit{short.wav} & \textit{long.wav} & \textit{noise.wav} & \textit{vivaldi.wav} \\
    \hline
    \multirow{2}{*}{\parbox{3.2cm}{\centering achieved with \\ \textit{spectrum\_search}}} & \multirow{2}{*}{\centering 1.549} & \multirow{2}{*}{\centering 2.414} & \multirow{2}{*}{\centering 2.569} & \multirow{2}{*}{\centering 2.5}  \\ 
    & & & &  \\ 
    \hline
    \multicolumn{1}{|c|}{possible}  & 1.551 & 2.436 & 2.594  & 2.526 \\ 
    \hline
    \multicolumn{1}{|c|}{required}  & 5.251 & 2.9321 & 6.219 & 5.748 \\ 
    \hline
    \end{tabular}
    \caption{Speedup achieved, possible, and required for all input files.}
    \label{speedup}
\end{table}

\vspace{0.3cm}

In more detail, the achieved speedup is calculated based on the \textit{TwoLAME total} execution times with \textit{spectrum\_search} and without \textit{Versat}, from table \ref{time}.
The possible speedup is calculated taking into account the proportion of the \textit{TwoLAME} that can be parallelized, which is obtained as the result of the division between \textit{psycho\_3\_threshold} and \textit{TwoLAME total} execution times without \textit{Versat}, also from table \ref{time}.
Lastly, the required speedup is calculated based on the \textit{TwoLAME total} execution times without \textit{Versat}, from table \ref{time}, and the Real-time, from table\ref{realtime}.

Focusing on the results, it is noticeable that the \textit{spectrum\_search} accelerator provides a speedup very close to the possible one (the maximum speedup achievable by accelerating \textit{psycho\_3\_threshold} function).
However, the required speedup indicates that the acceleration of \textit{psycho\_3\_threshold} is not enough to meet the real-time requirements. 
In particular, the first two input files encoding is approximately at 3.39 and 1.21 speedup from meeting the real-time requirements, respectively. In the first case, the input file has a small number of frames, which doesn't potentiate \textit{Versat} acceleration and, consequently, the achieved speedup stays far from the required one.
In the second case, the input file has a significant number of frames but the audio itself is simple, which implies a lower demand of \textit{psycho\_3\_threshold} function (executed by \textit{spectrum\_search} accelerator). For this reason, the required speedup is lower and easier to reach, which makes the achieved speedup very close to the required one.
The last two input files' encoding is approximately at 2.42 and 2.30 speedup from meeting the real-time requirements. These results give some guarantees about achieving the goal of this work since the third input file has a considerable number of frames and represents the worst-case scenario for real-time encoding (being a stereo white noise audio file). The fourth input file has a normal audio complexity since it is part of an existing soundtrack, and so the difference between achieved and required speedup is similar to the previous file.

Regarding the hardware architecture, the speedup values give a direct relation to the frequency necessary to meet the real-time requirements. Picking the \textit{noise.wav} input file as an example, the required speed up for IOB-MP2-E without hardware acceleration is 6.2. Since IOb-MP2-E operates at 100MHz, this means that an implementation for 620MHz would meet the goal. 

In a case where high clock speeds are a priority, application-specific integrated circuits (ASICs) are often more capable of running at faster frequencies. This integrated circuit can be optimized for specific tasks and, in some cases, achieve clock speeds as high as 500MHz in advanced semiconductor technologies like 28nm.
However, ASICs require a significant upfront investment, and the industry's low volume may make them less cost-effective. In contrast, FPGAs offer flexibility and programmability, allowing for quicker design iterations and so often preferred for low-volume or prototyping applications.
Decoders may benefit from ASICs, but for general-purpose or adaptable solutions, CPUs, GPUs, and FPGAs provide versatile off-the-shelf options. Although they may operate at slightly lower clock speeds, this brings the advantages of reduced power consumption, lower heat generation, and potentially longer component lifespan.
Considering this, the IOB-MP2-E has the potential to be a top-notch system in the future, as the achieved performance for the worst-case scenario shows that an implementation at 240MHz would meet the goal.