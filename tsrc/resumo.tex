O grupo de trabalho MPEG, estabelecido pela \textit{ISO} e \textit{IEC} em 1988, desempenhou um papel fundamental na transição de vídeo analógico para digital. A tecnologia MPEG suporta várias aplicações de transmissão, como DSS, DAB e DVB, com o MPEG-1 Audio sendo um formato de áudio com perdas globalmente popular. O padrão MPEG inclui três esquemas de codificação, com a Camada I para aplicações de baixa complexidade e a Camada II para cenários de um-para-muitos.
No desenvolvimento de System-on-Chip (SoC), equilibrar o desempenho e o custo é desafiador, especialmente na codificação de áudio. Esta tese apresenta um codificador de áudio MPEG-1/2 Camada II em ponto flutuante, oferecendo alta precisão ao custo de maior área de silício e consumo de energia em um mercado com núcleos comerciais limitados de MPEG-1/2 Camada II.
A arquitetura de software inclui a CPU \textit{VEXRISCV} em uma configuração IOb-SoC, usando o algoritmo de codificação \textit{TwoLAME}. Otimizações de software, incluindo simplificação de funções e redução de memória, são aplicadas. O perfil com um periférico TIMER destaca o impacto significativo da função \textit{psycho\_3\_threshold}, representando de 37\% a 64\% do tempo de execução, dependendo do arquivo de entrada.
No lado de hardware, o acelerador \textit{Versat} é integrado ao IOb-SoC para acelerar a função \textit{psycho\_3\_threshold}. Isso envolve a expansão da interface AXI para comunicação entre \textit{Versat}, RAM e a CPU. Dois aceleradores de hardware, \textit{spectrum\_search} e \textit{masking\_threshold}, abordam partes da função \textit{psycho\_3\_threshold}. A implementação em FPGA mostra um aumento no uso total de LUTs e LUTs lógicos, mas nenhuma alteração em blocos DSP, RAMB36, RAMB18 e URAM.
O êxito na portabilidade do \textit{TwoLAME} e nas otimizações de software destaca o impacto da utilização de recursos do \textit{Versat} dentro do FPGA. O tempo de execução é significativamente reduzido com o acelerador \textit{spectrum\_search} \textit{Versat}, aproximando-se dos requisitos de codificação em tempo real ao acelerar a função \textit{psycho\_3\_threshold}.
A comparação do aumento de velocidade alcançado com os requisitos de codificação em tempo real mostra uma estreita correlação para a maioria dos arquivos de entrada. O primeiro arquivo com menos quadros fica ligeiramente aquém, enquanto outros, especialmente o cenário de pior caso, apresentam resultados promissores.