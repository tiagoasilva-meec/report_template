\hspace{0.4cm} Este trabalho apresenta um codificador de áudio MPEG-1/2 Layer II para uma arquitetura incorporada RISC-V, com um acelerador de hardware reconfigurável. O software utiliza a biblioteca de código aberto TwoLAME, adaptada para o sistema. Embora tais sistemas sejam comuns em processadores incorporados comerciais, como o ARM, este trabalho é o primeiro a apresentar uma implementação RISC-V. A vantagem é que a arquitetura RISC-V é uma especificação aberta, com alguns designs de hardware de código aberto disponíveis. Um acelerador de hardware permite que o sistema funcione em ambientes de baixa frequência, como um dispositivo FPGA. O hardware do sistema é baseado no IOb-SoC, uma plataforma RISC-V de código aberto escrita em Verilog. O CPU VexRiscv foi escolhido, e o acelerador de hardware foi implementado usando a ferramenta de design de acelerador reconfigurável de código aberto Versat. O trabalho inclui otimizações de software e dois aceleradores de hardware para acelerar o cálculo do modelo psicoacústico do algoritmo. O desempenho base é 6,2 vezes mais lento do que o tempo real para um sistema a funcionar a 100MHz, o que indica que uma implementação a 620MHz atingiria o objetivo. Com aceleração de hardware, o desempenho alcançado é 2,4 vezes mais lento do que o tempo real para um sistema a funcionar a 100MHz, o que indica que uma implementação a 240MHz atingiria o objetivo.