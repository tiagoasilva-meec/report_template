The Moving Picture Experts Group (MPEG) has been pivotal in the shift from analog to digital audio. MPEG technology supports Digital Audio Broadcast (DAB), with MPEG-1 Audio being a globally popular lossy audio format. The MPEG standard includes coding schemes like Layer I for low complexity and Layer II for one-to-many applications.
In System-on-Chip (SoC) development, balancing performance and cost is challenging. This thesis introduces a floating-point MPEG-1/2 Layer II Audio encoder, offering high accuracy at the cost of increased silicon area in a market with limited commercial MPEG-1/2 Layer II Intellectual Property (IP) cores.
The software architecture features the \textit{VexRiscv} Central Processing Unit (CPU) in an IOb-SoC setup, using the \textit{TwoLAME} encoding algorithm. Software optimizations are applied. Profiling with a \textit{TIMER} peripheral highlights the significant impact of the \textit{psycho\_3\_threshold} function, accounting for 37\% to 64\% of execution time.
On the hardware side, the \textit{Versat} accelerator is integrated into IOb-SoC to speed up the \textit{psycho\_3\_threshold} function. This involves expanding the Advanced eXtensible Interface (AXI) interface for communication. Two hardware accelerators, \textit{spectrum\_search} and \textit{masking\_threshold}, address parts of the \textit{psycho\_3\_threshold} function. The field-programmable gate array (FPGA) implementation shows increased Total Look Up Tables (LUTs), LUTRAMs, Flip-Flops, and Digital Signal Processor (DSP) Blocks.
The execution time of \textit{psycho\_3\_threshold} function is significantly reduced with the \textit{spectrum\_search} \textit{Versat} accelerator, approaching real-time encoding requirements.
Comparing the achieved speedup with real-time encoding requirements shows that the first file with fewer frames falls slightly short, while others, especially the worst-case scenario, exhibit promising outcomes.