%The Abstract is a summary of the work written for specialists. Its ideal size is one-third of a page, and it should quickly explain the problem, the current work, the proposed solution, the results obtained, and why they improve the state-of-art.

%MAX 250 words

Since its establishment in 1988, the Moving Picture Experts Group (MPEG) has made an indelible mark on the transition from analog to digital video, especially in broadcasting, storage, multimedia, and telecommunication fields.
Today, MPEG-1 Audio is the most widely compatible lossy audio format in the world, under the International Organization for Standardization/International Electrotechnical Commission (ISO/IEC) 11172-3 standard.

Apart from chips, like \textit{CX23415 MPEG-2 Codec}, \textit{MPEG-2 Encoder CW-4888} and \textit{Futura II ASI+IP}, there are only two IP cores specifically for MPEG-1/2 Layer I/II Audio on the market, namely the \textit{CWda74} and \textit{IPB-MPEG-SE}.
Besides that, an IP core allows the developer to design a top-notch encoder, with less area/volume, weight, and power consumption than the ones available on the market.
%Besides that, the importance of IP design is growing, as consumer demands require quick product development. Also, an IP design is usually proven in a fully-tested product before being licensed, avoiding risks in System-on-Chip (SoC) design. Last, an IP design is provided in a hardware description language, allowing easier implementation in both field-programmable gate array (FPGA) and application-specific integrated circuit (ASIC).\\
Therefore, this work proposes an intellectual property (IP) core capable of encoding MPEG-1/2 Layer I/II Audio. 

The IP will be developed using \textit{IObundle, Lda}'s IOb-SoC, a System-on-Chip template comprising an open-source RISC-V processor. The TwoLAME GitHub repository, an optimized MPEG Audio Layer 2 encoding software, will provide the algorithm.
