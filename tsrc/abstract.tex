The Moving Picture Experts Group (MPEG), established by International Organization for Standardization (ISO) and International Electrotechnical Commission (IEC) in 1988, has been pivotal in the shift from analog to digital video. MPEG technology supports various broadcasting applications, such as Digital Satellite System (DSS), Digital Audio Broadcast (DAB), and Digital Video Broadcast (DVB), with MPEG-1 Audio being a globally popular lossy audio format. The MPEG standard includes three coding schemes, with Layer I for low-complexity applications and Layer II for one-to-many scenarios.
In System-on-Chip (SoC) development, balancing performance and cost is challenging, especially in audio encoding. This thesis introduces a floating-point MPEG-1/2 Layer II Audio encoder, offering high accuracy at the cost of increased silicon area and power consumption in a market with limited commercial MPEG-1/2 Layer II Intellectual Property (IP) cores.
The software architecture features the \textit{VEXRISCV} Central Processing Unit (CPU) in an IOb-SoC setup, using the \textit{TwoLAME} encoding algorithm. Software optimizations, including function simplification and memory reduction, are applied. Profiling with a \textit{TIMER} peripheral highlights the significant impact of the \textit{psycho\_3\_threshold} function, accounting for 37\% to 64\% of execution time, depending on the input file.
On the hardware side, the \textit{Versat} accelerator is integrated into IOb-SoC to speed up the \textit{psycho\_3\_threshold} function. This involves expanding the Advanced eXtensible Interface (AXI) interface for communication between \textit{Versat}, Random Access Memory (RAM), and the CPU. Two hardware accelerators, \textit{spectrum\_search} and \textit{masking\_threshold}, address parts of the \textit{psycho\_3\_threshold} function. The field-programmable gate array (FPGA) implementation shows increased Total Look Up Tables (LUTs) and Logic LUTs usage but no change in Digital Signal Processor (DSP) Blocks, RAMB36, RAMB18, and URAM.
The successful porting of \textit{TwoLAME} and software optimizations highlight the resource utilization impact of \textit{Versat} within the FPGA. Execution time is significantly reduced with the \textit{spectrum\_search} \textit{Versat} accelerator, approaching real-time encoding requirements by accelerating the \textit{psycho\_3\_threshold} function.
Comparing the achieved speedup with real-time encoding requirements shows close alignment for most input files. The first file with fewer frames falls slightly short, while others, especially the worst-case scenario, exhibit promising outcomes.

%This thesis presents a comprehensive exploration of enhancing MPEG-1/2 Layer II audio encoding, leveraging floating-point processing and hardware acceleration, thereby contributing to the diversification of available solutions in a market currently dominated by a single product.